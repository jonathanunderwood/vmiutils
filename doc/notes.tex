\documentclass{article}
\usepackage{amsmath}
\renewcommand{\eqref}[1]{Eq.~(\ref{#1})}
\renewcommand{\exp}[1]{\ensuremath{\mathrm{e}^{#1}}}
\newcommand{\pd}[2]{\ensuremath{\frac{\partial#1}{\partial#2}}}
\newcommand{\dee}{\ensuremath{\mathrm{d}}}
\title{Notes on inversion algorithms}
\author {Jonathan G. Underwood}

\begin{document}
\maketitle
\tableofcontents

\section{Coordinate systems}
We adopt the convention that the observed image $F(x,z)$ lies in the
$xz$-plane, with $z$ designated as the axis of cylindrical symmetry. We also
wish to express the image in polar coordinates, $F(R, \Theta)$ such that
\begin{subequations}
  \begin{align}
    z&=R\cos\Theta\\
    x&=R\sin\Theta\\
    R^2&=x^2+z^2
  \end{align}
\end{subequations}
The distribution of photofragments is most naturally expressed in polar
coordinates, $f(r, \theta, \phi)\equiv f(r, \theta)$. We assume a
cylindrically symmetric distribution of fragments. We choose a system of
coordinates such that the $x$ and $z$ directions are coincident to that used
for the image plane.
\begin{subequations}
  \begin{align}
    z&=r\cos\theta\\
    x&=r\sin\theta\cos\phi\\
    y&=r\sin\theta\sin\phi\\
    r^2&=x^2+y^2+z^2
  \end{align}
\end{subequations}
It is useful to define the quantity $\rho$ representing the distance from the
tip of the vector $\mathbf{r}$ to the $z$-axis
\begin{subequations}
  \begin{align}
    \rho&=r\sin\theta\\
    \rho^2&=x^2+y^2
  \end{align}
\end{subequations}
We also note the following relationships
\begin{subequations}
  \begin{align}
    r^2&=R^2+y^2\\
    \rho^2&=r^2-z^2\\
    R&=\frac{r\cos\theta}{\cos\Theta}\\
    \tan\Theta&=\frac{x}{z}\\&=\tan\theta\cos\phi
  \end{align}
\end{subequations}

\section{Forward Abel integral expressed in Cartesian coordinates}
We can express the image $F(x,z)$ in terms of the distribution expressed in
cylindrical polar coordinates $f(\rho,z\, \phi)\equiv f(\rho, z)$ as 
\begin{equation}
  F(x,z)=\int_{-\infty}^{\infty}
  f(\rho, z)\;\dee{y}.
\end{equation}
Since $y=\sqrt{\rho^2-x^2}$,
\begin{equation}
  \frac{\dee{y}}{\dee{\rho}}
  =\frac{\rho}{\sqrt{\rho^2-x^2}}
\end{equation}
and so
\begin{equation}
  \label{eq:abel_cartesian}
  F(x,z)=2\int_{|x|}^{\infty}
  \frac{\rho f(\rho, z)}{\sqrt{\rho^2-x^2}}
  \;\dee\rho.
\end{equation}

\section{Forward Abel integral expressed in spherical polar coordinates} 
We would like to express the observed image $F(R, \Theta)$ in terms of the
distribution $f(r, \theta)$. The Abel integral is
\begin{equation}
  F(R, \Theta)=2\int_{|x|}^\infty
  \frac{\rho f(r, \theta)}{\sqrt{\rho^2-x^2}}\;\dee \rho.
\end{equation}
with $x=R\sin\Theta$ and $\rho=r\sin\theta$. Since $\rho=\sqrt{r^2-z^2}$,
\begin{equation}
  \frac{\dee \rho}{\dee r}=
  \frac{r}{\sqrt{r^2-z^2}},
\end{equation}
such that
\begin{equation}
  \rho\;\dee \rho=r\;\dee r
\end{equation}
and since $r=\sqrt{\rho^2+z^2}$
\begin{equation}
  F(R, \Theta)=
  2\int_{\sqrt{|x|^2+z^2}}^\infty
  \frac{rf(r, \theta)}{\sqrt{r^2-z^2-x^2}}\;\dee r
\end{equation}
which can be re-expressed as
\begin{equation}
  \label{eq:abel_polar}
  F(R, \Theta)=
  2\int_{R}^\infty
  \frac{rf(r, \theta)}{\sqrt{r^2-R^2}}\;\dee r.
\end{equation}
with 
\begin{equation}
  \theta=\arccos\left(
    \frac{R}{r}\cos\Theta
  \right)
\end{equation}
This equation is analogous to \eqref{eq:abel_cartesian} with the equivalence
$r\leftrightarrow\rho$ and $R\leftrightarrow x$. However, the equivalence
$\Theta\leftrightarrow z$ is not quite so straightforward.

It is readily seen that the inverse transform of \eqref{eq:abel_polar}
diverges as $r\rightarrow0$. This has the consequence of placing inversion
noise at the centre of the inverted image, rather than along the $z$-axis
(i.e. as $x\rightarrow0$) as is the case of the inverse of
\eqref{eq:abel_cartesian}.

\section{Implementation notes for pbasex}
We use a slightly modified implementation of pbasex, using
\eqref{eq:abel_polar}. Following GNP we expand the photofragment distribution
as
\begin{equation}
  f(r,\theta)=
  \sum_{k=0}^{k_\mathrm{max}}
  \sum_{l=0}^{l_\mathrm{max}}
  c_{kl}f_{kl}(r,\theta)
\end{equation}
In GLP, the basis functions $f_{kl}(r,\theta)$ are defined as
\begin{equation}
  f_{kl}(r,\theta)=
  \exp{-(r-r_k)^2/\sigma'}P_l(\cos\theta).
\end{equation}
which departs from the standard definition of a Gaussian with $\sigma'$ in the
exponent rather than $2\sigma^2$. In our implementation we will use the more
standard definition
\begin{equation}
  f_{kl}(r,\theta)=
  \exp{-(r-r_k)^2/2\sigma^2}P_l(\cos\theta).
\end{equation}
where the FWHM of the Gaussian radial function is given by
$\mathrm{FWHM}=2\sigma\sqrt{2\ln2}$. The equivalence is clearly
$\sigma'=2\sigma^2$.

The image can then be expressed as
\begin{equation}
  F(R, \Theta)=
  \sum_{k=0}^{k_\mathrm{max}}
  \sum_{l=0}^{l_\mathrm{max}}
  c_{kl}g_{kl}(R,\Theta)
\end{equation}
with
\begin{equation}
  g_{kl}(R,\Theta)=
  2\int_{R}^\infty
  \frac{rf_{kl}(r, \theta)}{\sqrt{r^2-R^2}}\;\dee r.
\end{equation}

The observed image is actually sampled on a grid of cartesian pixels which is
the converted to a regular grid of $(R, \Theta)$ pixels.
\begin{equation}
  F_{ij}(R, \Theta)=
  \sum_{k=0}^{k_\mathrm{max}}
  \sum_{l=0}^{l_\mathrm{max}}
  c_{kl}g_{ij;kl}(R,\Theta)
\end{equation}
where
\begin{equation}
  g_{ij;kl}(R,\Theta)=
  2
  \int_{i\Delta_R}^{(i+1)\Delta_R}\dee R
  \int_{j\Delta_\Theta}^{(j+1)\Delta_\Theta}\sin\Theta\;\dee \Theta
  \int_{R}^\infty
  \frac{rf_{kl}(r, \theta)}{\sqrt{r^2-R^2}}\;\dee r.
\end{equation}

The GLP implementation first re-bins the image into a $256\times256$ grid of
$(R, \Theta)$ pixels. By default, 128 radial basis functions are then used,
$r_k$ is chosen to be $2k$, and $\sigma'$ is chosen to be 2 i.e. $\sigma=1$
and $\mathrm{FWHM}=2.354$. In other words, the FWHM is chosen to be
approximately equal to the basis function radial separation. We will adopt a
similar strategy, setting $\mathrm{FWHM}=w$, where $w$ is the basis function
radial separation given in terms of the number of radial basis functions $N_k$
and maximum radial distance (in pixels) as $w=r_\mathrm{max}/N_k$.

\section{Inclusion of a detection function in the Pbasex method}
For a single value of $r$, we can write
\begin{equation}
  \label{eq:Ff}
  F(R, \Theta)S_{R\Theta}=f(r, \theta)S_{\theta\phi},
\end{equation}
where $S_{R\Theta}$ is the elementary surface on the plane of the detector and
$S_{theta\phi}$ is the elementary surface on the sphere in the initial
coordinate system:
\begin{gather}
  S_{R\Theta} = R\;\dee R\;\dee\Theta,\\
  \label{eq:Sthetaphi}
  S_{\theta\phi} = r^2\sin\theta\;\dee\theta\;\dee\phi.
\end{gather}
$S_{R\Theta}$ may be re-written as 
\begin{equation}
  \label{eq:SRTheta}
  \dee R\;\dee\Theta=|\mathbf{J}|\;\dee\theta\;\dee\phi,
\end{equation}
where the determinant of the Jacobian $\mathbf{J}$ is given by
\begin{equation}
  \label{eq:Jacobian}
  |\mathbf{J}| = 
  \left|
    \pd{R}{\theta}\pd{\Theta}{\phi}
    -
    \pd{\Theta}{\theta}\pd{R}{\phi}
    \right|.
\end{equation}
Substituting \eqref{eq:Sthetaphi}, \eqref{eq:SRTheta} and \eqref{eq:Jacobian}
into \eqref{eq:Ff} we can write
\begin{equation}
  F(R, \Theta)=
  \frac{f(r,\theta)r^2\sin\theta}{R|\mathbf{J}|}.
\end{equation}
Noting that
\begin{subequations}
  \begin{align}
    R&=r\sqrt{\cos^2\theta+\sin^2\theta\sin^2\phi}\\
    \Theta&=\arctan(\sin\phi\tan\theta)
  \end{align}
\end{subequations}
we can write the Jacobian as
\begin{equation}
  J = -\frac{2 r \sin ^2(\theta ) \cos (\phi )}
  {\sqrt{2 \cos (2 \theta ) \cos ^2(\phi )-\cos (2 \phi )+3}}
\end{equation}

\end{document}
%%% Local Variables: 
%%% mode: latex
%%% TeX-master: t
%%% End: 
